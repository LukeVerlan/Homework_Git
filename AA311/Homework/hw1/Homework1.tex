 
\documentclass[12pt]{article}
 
\usepackage[margin=1in]{geometry}
\usepackage{amsmath,amsthm,amssymb,mathrsfs,graphicx,colortbl,xspace}

\newcommand{\N}{\mathbb{N}}
\newcommand{\Z}{\mathbb{Z}}
\newcommand{\Q}{\mathbb{Q}}
\newcommand{\R}{\mathbb{R}}
\newcommand{\B}{\mathcal{B}}
\newcommand{\vepsilon}{\varepsilon}
\newcommand{\bigspace}{\;\;\;\;\;\;\;\;\;\;\;\;\;\;\;\;\;\;\;\;\;\;}
 
\newenvironment{theorem}[2][Theorem]{\begin{trivlist}
\item[\hskip \labelsep {\bfseries #1}\hskip \labelsep {\bfseries #2.}]}{\end{trivlist}}
\newenvironment{lemma}[2][Lemma]{\begin{trivlist}
\item[\hskip \labelsep {\bfseries #1}\hskip \labelsep {\bfseries #2.}]}{\end{trivlist}}
\newenvironment{exercise}[2][Exercise]{\begin{trivlist}
\item[\hskip \labelsep {\bfseries #1}\hskip \labelsep {\bfseries #2.}]}{\end{trivlist}}
\newenvironment{problem}[2][Problem]{\begin{trivlist}
\item[\hskip \labelsep {\bfseries #1}\hskip \labelsep {\bfseries #2.}]}{\end{trivlist}}
\newenvironment{question}[2][Question]{\begin{trivlist}
\item[\hskip \labelsep {\bfseries #1}\hskip \labelsep {\bfseries #2.}]}{\end{trivlist}}
\newenvironment{corollary}[2][Corollary]{\begin{trivlist}
\item[\hskip \labelsep {\bfseries #1}\hskip \labelsep {\bfseries #2.}]}{\end{trivlist}}
\usepackage{enumitem}
\usepackage{tikz}
\newcommand{\Cross}{\mathbin{\tikz [x=1.4ex,y=1.4ex,line width=.2ex] \draw (0,0) -- (1,1) (0,1) -- (1,0);}}%
\newenvironment{solution}{\begin{proof}[Solution]}{\end{proof}}
 
\begin{document}
 
% --------------------------------------------------------------
%                         Start here
% --------------------------------------------------------------
 
\title{Homework 1}%replace X with the appropriate number
\author{Luke Verlangieri\\ %replace with your name
} %if necessary, replace with your course title
 
\maketitle

\begin{exercise}{2.1} %You can use theorem, exercise, problem, or question here.  Modify x.yz to be whatever number you are proving
Consider the low-speed flight of the Space Shuttle as it is nearing a landing. 
If the air pressure and temperature at the nose of the shuttle are 
$1.2 \, \text{atm}$ and $300 \, \text{K}$, respectively, 
what are the density and specific volume?
\end{exercise}

\begin{solution}
\;\newline\newline
\textbf{Given:}
\begin{center}
    P = 1.2 atm, T = 300 K
\end{center}
\textbf{Find:}
\[
    \rho, \nu
\]
\textbf{Properties:}
\[
    P = \rho RT, \nu = \frac{1}{\rho}, R_{air} = 287 \frac{\text{J}}{\text{kg K}},
\]
\begin{center}
    1 atm $\approx$ 101325 Pa
\end{center}
\textbf{Analysis:} \par \; \newline \indent
Start by converting to SI units: 1.2 atm = 121590 Pa. Consulting the ideal gas law, 
$P\nu = RT$ therefore 
\[
    \nu = \frac{RT}{P} = \frac{(300 \space \text{K}) (287 \space \frac{\text{J}}{\text{kg K}})}{(121590\space\text{Pa})} 
    \approx 0.708 ~\mathrm{m^3/kg}
\]
and finally
\[
    \rho = \frac{1}{\nu} = \frac{1}{0.708~\mathrm{m^3/kg}} \approx 1.412~\mathrm{kg/m^3}
\]
\end{solution}

\newpage

\begin{exercise}{2.9}
Consider a flat surface in an aerodynamic flow (say a flat sidewall of a 
wind tunnel). The dimensions of this surface are \(3 \, \text{ft}\) in the 
flow direction (the \(x\) direction) and \(1 \, \text{ft}\) perpendicular to 
the flow direction (the \(y\) direction). Assume that the pressure 
distribution (in pounds per square foot) is given by 
\(p = 2116 - 10x\) and is independent of \(y\). Assume also that the shear 
stress distribution (in pounds per square foot) is given by 
\(\tau_w = \tfrac{90}{\sqrt{x+9}}\) and is independent of \(y\) as shown in 
the figure below. In these expressions, \(x\) is in feet, and \(x=0\) at the 
front of the surface. Calculate the magnitude and direction of the net 
aerodynamic force on the surface.
\end{exercise}

\begin{solution}
    \;\newline\newline
    \textbf{Given:}
    \begin{center}
        \[ 
            \tau_w(x) = \frac{90}{\sqrt{x + 9}}, P(x) = 2116 - 10x, L = 3\text{ft}, W = 1\text{ft}
        \]
    \end{center}
    \textbf{Find:}
    \begin{center}
        aerodynamic force acting on the surface: magnitude \& direction
    \end{center}
    \textbf{Assumptions:}
    \begin{center}
        The surface is entirely flat \& no forces act in the y direction
    \end{center}
    \textbf{Properties:}
    \begin{center}
        \[
            F = \iint Pn\space da,  A = L*W
        \]
    \end{center}
    \textbf{Analysis:} \par \; \newline
        Firstly, we can compute the force applied to the surface from both the pressure and shear stress on the surface.
        The pressure acts normal to the surface, therefore the force acting on the surface is also normal to the surface (-z direction).
        The shear stress acts in the positive x direction. The magnitude of force applied by the pressure is therefore:
        \[ 
            F_p = \int_{0}^{L}\int_{0}^{W}P(x) \cdot da = \int_{0}^{3}(2116 - 10x)\; dx = 6303 \; \text{lbs}
        \]
        its direction mirrors that of the pressure, being the -z direction. Next we can analyze the shear stress on the surface:
        \[ 
            F_{\tau} = \int_{0}^{L}\int_{0}^{W}\tau(x) \cdot da = \int_{0}^{3}\frac{90}{\sqrt{x + 9}} \; dx = 83.54 \; \text{lbs}
        \]
        The direction of this force is in the same direction as the flow/shear stress.\newline
        Therefore the magnitude of the total aerodynamic force is given by: 
        \[
            ||F_{aerodynamic}|| = \sqrt{F_{\tau}^2 + F_p^2} \approx 6304 \; \text{lbs} 
        \]
        Breaking this into a vector
        \[  
            F_{aerodynamic} = 
            \begin{bmatrix}
                83.54 \\
                0 \\
                -6303
            \end{bmatrix}, \; 
            \boldsymbol{\hat{e}} = \frac{F_{aerodynamic}}{||F_{aerodynamic}||} = 
            \begin{bmatrix}
                0.013\\
                0\\
                -0.987
            \end{bmatrix}
        \]
\textbf{Results:}
\begin{center}
    \[ ||F_{aerodynamic}|| \approx 6304 \; \text{lbs} \;\& \;\boldsymbol{\hat{e}} = 
        \begin{bmatrix}
            0.013\\
            0\\
            -0.987
        \end{bmatrix}
    \]
\end{center}
\textbf{Comments:} \; \newline \newline 
    The pressure causes significantly more stress on the surface than the shear stress. I assume this is because we assume
    that the flow is only the x direction and that the surface only exists in the xy plane. Once we have a surface that is rotated 
    out of this plane the flow exists in I expect to see a stronger shear force on the surface.
\end{solution}
\pagebreak
\begin{exercise}{2.14}
    In  a  gas  turbine  jet  engine, the  pressure of  the  incoming air  is increased by flowing through a compressor; 
    the air then enters a combusto rthat looks vaguely like a long can (sometimes called the combustion can). 
    Fuel is injected into the combustor and burns with the air, and then the burned fuel–air mixture exits the combustor at a higher 
    temperature than  the air coming into the combustor. (Gas turbine jet engines are discussed in Ch. 9).
    The pressure of the flow through the combustor remains relatively constant; that is, the combustion process is at constant pressure. 
    Considerthe case where the gas pressure and temperature entering the combustor are $4 \cdot 10^6 \frac{\text{N}}{\text{m}^2} $ and 900 K, 
    respectively, and the gas temperature exiting the combustor is  1500 K.  
    Calculate the  gas  density at (a)  the  inlet  to  the combustor and (b) the exit of the combustor.
    Assume that the specific gas constant for the fuel–air mixture is the same as that for pure air.
\end{exercise}
\begin{solution} \; \newline \newline
    \textbf{Given:}
    \begin{center}
        \[ P = const = 4 \cdot 10^6 \frac{\text{N}}{\text{m}^2}, \; T_{inlet} = 900 \; \text{K}, \; T_{outlet} = 1500 \; \text{K}, \; R = R_{air} = 287 \frac{\text{J}}{\text{kg $\cdot$ K}} \]
    \end{center}
    \textbf{Find:}
        \[ \rho_{inlet} \; \& \; \rho_{outlet} \]
    \textbf{Properties:}
        \[ P = \rho * RT \] 
    \textbf{Analysis:}
        \; \newline\newline\indent We can rewrite this equation $ P = \rho * RT $ to be $ \rho = \frac{P}{RT} $. We can then solve for the inlet and outlet density by simply plugging in values.
        \[ \rho_{inlet} = \frac{(4 \cdot 10^6 \; \text{Pa})}{(287 \; \frac{\text{J}}{\text{kg $\cdot$ K}})(900 \; \text{K})} = 15.49 \; \frac{\text{kg}}{\text{m}^3}
        \quad \& \quad 
        \rho_{outlet} = \frac{(4 \cdot 10^6 \; \text{Pa})}{(287 \; \frac{\text{J}}{\text{kg $\cdot$ K}})(1500 \; \text{K})} = 9.29 \; \frac{\text{kg}}{\text{m}^3}
        \]
\end{solution}


%Note 1: The * tells LaTeX not to number the lines.  If you remove the *, be sure to remove it below, too.
%Note 2: Inside the align environment, you do not want to use $-signs.  The reason for this is that this is already a math environment. This is why we have to include \text{} around any text inside the align environment.

% --------------------------------------------------------------
%     You don't have to mess with anything below this line.
% --------------------------------------------------------------
 
\end{document}