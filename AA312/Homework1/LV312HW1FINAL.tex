\documentclass[12pt]{article}
\usepackage[left=1in,top=1in,right=1in,bottom=1in]{geometry}

\pagenumbering{arabic}
\usepackage{hyperref}
\usepackage{graphicx}
\usepackage{color}
\usepackage{amsmath,amssymb,amsthm}
\usepackage{cite,color,comment,xspace}
\usepackage{caption}
\captionsetup[figure]{font=small}
\usepackage{epstopdf}
\usepackage{color}
\usepackage{titlesec}
\usepackage{rotating}
\usepackage{wrapfig}
\usepackage{fancyhdr}
\pagestyle{fancy}
\usepackage{enumitem}
\usepackage{subcaption}
\usepackage{multicol}
\usepackage[square,numbers]{natbib}
\setlength{\bibsep}{0pt plus 0.3ex}
\usepackage[skip=2pt]{caption} % example skip set to 2pt
\setlength{\belowcaptionskip}{-2mm} % Chosen fairly arbitrarily
\usepackage{array}
\newcolumntype{x}[1]{>{\centering\arraybackslash\hspace{0pt}}p{#1}}

\definecolor{darkRed}{rgb}{0.2,0,0.44} 


\newcommand*{\vertbar}{\rule[-1ex]{0.5pt}{2.5ex}}
\newcommand*{\horzbar}{\rule[.5ex]{2.5ex}{0.5pt}}

\newcommand{\workingoutspace}[1]{\vspace{#1cm}}
\newcommand{\course}{AA312}
\newcommand{\instructor}{Leung}
\newcommand{\quarteryear}{Winter 2026}
\newcommand{\shorttitle}{\course}
\newcommand{\homeworknumber}{01}
\newcommand{\Lagr}{\mathcal{L}}


\title{\vspace{-2.5truecm}  \bf \course: Structural Vibrations \vspace{-0.5Truecm}
}

\author{\quarteryear  $\quad\mid\quad$ Homework \homeworknumber\\
{\small \textbf{Module 1}: Fundamentals of LTI systems}\\
{\small \textbf{Due Date}: Friday Jan 23th 11:59PM. Submit via Canvas.}\\
}

\date{} 


\lfoot{\quarteryear}
\lhead{\shorttitle}
\rhead{Homework \homeworknumber}
\rfoot{Instructor: \instructor}


% \renewcommand{\baselinestretch}{0.94}	

%%%%%%%%%%%%%%%%%%%%%%%%%%%%%%%%%%

\begin{document}
\maketitle 

\vspace{-.8cm}

\noindent \textit{As a student of The University of Washington, I shall abide by the University’s Student Conduct Code.}\\

\noindent \text{Name and Signature}: Luke Verlangieri \:{\hspace{12cm}}

\vspace{4mm}

\begin{flushleft}
\phantom{a}
\end{flushleft}
 \vspace{-1.7truecm}
{\color{darkRed}\rule{\textwidth}{0.05in}}
\thispagestyle{empty}

\noindent \textbf{Short response problems.} See Canvas Quiz for the short response problems. They are due at the same time as these long-response problems.\\


\noindent \textbf{Long response problems.} Answer the following long response problems below. Please keep your work tidy and legible. Points cannot be awarded for disorganized and illegible work.

\begin{enumerate}
    \item 
    Given two real-world examples of mechanical dynamical systems that could be modeled with mass(es), spring(s), and damper(s). Provide two diagrams for each system: one illustrating the system itself (e.g., a photo or simplified diagram) and another where you approximate it as a mass-spring-damper system.

    \begin{figure}[h]
      \centering
      \includegraphics[width=0.6\textwidth]{figs/Problem1.png}
      \caption{Car suspension and Pogo Stick MSD Diagrams}
      \label{fig:1}
    \end{figure}

    \newpage
    \item \textbf{(Solving ODEs using Laplace Transform.)} Solve the following ODEs by taking the Laplace transform, rearrange to get an expression for $X(s)$, and then take the inverse Laplace transform of $X(s)$. 
    \begin{enumerate}[label=(\alph*)]
        \item $\ddot{x}(t) + 3\dot{x}(t) + 2x(t) = 0,\: x(0) = 3,\; \dot{x}(0) = 1$ \\
        \[ 
          \Lagr\left\{\ddot{x}(t)\right\} + 3\Lagr\left\{\dot{x}(t)\right\} + 2\Lagr\left\{x(t)\right\} = 0 
        \]
        \\
        \[
          \left( s^2X(s) - sx(0) - \dot{x}(0) \right) + 3\left(sX(s) - x(0)\right) + 2X(s) = 0
        \]
        \\
        \[
          X(s)\left(s^2 + 3s + 2\right) = x(0)\left(s+3\right) + \dot{x}(0) = 3s + 10
        \]
        \\
        \[
          X(s) = \frac{3s + 10}{s^2 + 3s + 2} = \frac{3s + 10}{(s+2)(s+1)}
        \]

        Take Partial Fraction Decomposition
        \[ 
          3s + 10 = A(s+1) + B(s+2) 
        \]
        Try roots $s = -1, -2$
        \[ 
          s = -2 \longrightarrow A = -4
        \]
        \[
          s = -1 \longrightarrow B = 7
        \]
        Plug back in to original equation
        \[
          X(s) = -\frac{4}{s+2} + \frac{7}{s+1}
        \]
        Take inverse Laplace 
        \[
          x(t) = -4\Lagr^{-1}\left(\frac{1}{s+2}\right) + 7\Lagr^{-1}\left(\frac{1}{s+1}\right) \\
          \Longrightarrow x(t) = -4e^{-2t} + 7e^{-t}
        \] 

        \pagebreak

        \item $\ddot{x}(t) + 4\dot{x}(t) + 5x(t) = u(t)$ where $u(t) = \delta(t)$ is the Dirac delta function, i.e., impulse. Assume zero initial conditions (i.e. $x(0) = 0,\; \dot{x}(0) = 0)$. \\
        
        Note: $\Lagr\left\{\delta(t)\right\} = 1$ \\

        \[
          \Lagr\left\{\ddot{x}(t)\right\} + 4\Lagr\left\{\dot{x}(t)\right\} + 5\Lagr\left\{x(t)\right\} = \Lagr\left\{\delta(t)\right\}
        \]
        \\ 
        \[
          \left(s^2X(s) - sx(0) - \dot{x}(0)\right) + 4\left(sX(s) - x(0)\right) + 5X(s) = 1
        \]
        \\
        Plug in given boundary conditions and factor
        \[
          X(s)\left(s^2 + 4s + 5\right) = 1
        \]
        \\
        \[
          X(s) = \frac{1}{s^2+4s+5} = \frac{1}{(s+2)^2 + 1}
        \]
        \\
        Apply Frequency shift (a = -2)
        \[  
          x(t) = e^{-2t}\Lagr^{-1}\left\{\frac{1}{s^2 + 1}\right\} = e^{-2t}\sin(t)
        \]
        \\
        \[  
          \Longrightarrow x(t) = e^{-2t}\sin(t)
        \]

        \pagebreak

        \item Suppose that the solution to the system $\ddot{x}(t) + 4\dot{x}(t) + 5x(t) = 0,\; x(0) = 1,\; \dot{x}(0) = 2$ is $x_\mathrm{init}(t)$. Now, if the system has initial conditions $x(0) = 1,\; \dot{x}(0) = 2$ and also has an impulse input applied at the same initial time, what is the expression for the resulting output? Justify your answer. \\
        
        Take the Laplace transform of the function, with an impulse input $\delta(t)$
        \[
          \Lagr\left\{\ddot{x}(t)\right\} + 4\Lagr\left\{\dot{x}(t)\right\} + 5\Lagr\left\{x(t)\right\} = \Lagr\left\{\delta(t)\right\}
        \] 
        \\
        Factor and solve for X(s)
        \[
          \left(s^2X(s) - sx(0) - \dot{x}(0)\right) + 4\left(sX(s) - x(0)\right) + 5X(s) = 1
        \]
        \\
        \[  
          X(s) = \frac{x(0)(s+4) + \dot{x}(0) + 1}{s^2 + 4s + 5}
        \]
        \\
        From here the solution can be split up into two parts. the $ +1 $ is a direct result of the impulse response. Therefore we can separate the impulse response term from the no input term in order to get the superposition $X(s) = X_{init}(s) + X_{impulse}(s)$
        \[
          X(s) = \frac{x(0)(s+4) + \dot{x}(0)}{s^2 + 4s + 5} + \frac{1}{s^2 + 4s + 5} = X_{init}(s) + X_{impulse}(s) 
        \]
        \\
        Then from the previous question, we know the inverse laplace of $ \frac{1}{s^2 + 4s + 5} $, Hence
        \[
          \Longrightarrow x(t) = x_{init}(t) + e^{-2t}\sin(t)
        \]
    \end{enumerate}
    \newpage


    \newpage
    \item \textbf{(Transfer function and system properties.)} Write down the transfer function of the following systems where $x(t)$ is the output, and $u(t)$ is the input. Also, write down their corresponding poles/zeros and whether they are BIBO stable. Provide your working below.
    \begin{enumerate}[label=(\alph*)]
        \item $\ddot{x}(t) + \dot{x}(t) - 6x(t) = u(t)$

        \begin{table}[h]
          \centering
            \begin{tabular}{cc}
              Transfer function:  $\frac{1}{(s+3)(s-2)}$ & \\
              Pole(s):  $2, -3$ &  \\
              Zero(s):   None &  \\
              BIBO stable: No & \\
            \end{tabular}
        \end{table}

        \[
          \Lagr\left\{\ddot{x}(t)\right\} + \Lagr\left\{\dot{x}(t)\right\} - 6\Lagr\left\{x(t)\right\} = \Lagr\left\{u(t)\right\}
        \]
        \[
          s^2X(s) + sX(s) - 6X(s) = U(s)
        \]
        \[
          \frac{X(s)}{U(s)} = \frac{1}{s^2 + s - 6} = \frac{1}{(s+3)(s-2)}
        \]
        \\
        The roots to the denominator are -3, and 2. Because 2 is a positive real number, this is not BIBO stable. \\ 
        
        \item $\ddot{x}(t) + 4\dot{x}(t) + 29x(t) = u(t)$ 

        \begin{table}[h]
            \centering
            \begin{tabular}{cc}
              Transfer function: $ \frac{1}{s^2 + 4s  + 29} $ & \\
              Pole(s): $-2 + 5i, -2 - 5i$  &  \\
              Zero(s):   None &  \\
              BIBO stable: Yes  &  \\
            \end{tabular}
        \end{table}
        
        \[ \Lagr\left\{\ddot{x}(t)\right\} + 4\Lagr\left\{\dot{x}(t)\right\} + 29\Lagr\left\{x(t)\right\} = \Lagr\left\{u(t)\right\} \]
        \[ s^2X(s) + 4sX(s) + 29X(s) = U(s) \]
        \[ \frac{X(s)}{U(s)} = \frac{1}{s^2 + 4s  + 29} \] 
        \\ 
        Use the quadratic formula to solve for the roots
        \[ \frac{(-4)}{2} \pm \frac{\sqrt{16 - 116}}{2} \Longrightarrow -2 \pm 5i \]
        \\
        All roots have negative real parts, therefore this is BIBO stable

    \end{enumerate}

    
    \newpage
    \item \textbf{(Modeling and natural frequency.)}
    A control pedal of an aircraft can be modeled as a single-degree-of-freedom system shown in Figure~\ref{fig:pedal}. Consider the lever as a massless shaft and the pedal at the bottom as a lumped mass at the end of the shaft (that is, ignore the exact shape of the pedal and treat it like a point mass).
    Assume the spring to be unstretched at $\theta = 0$, acceleration due to gravity $g$ is pointing down. \textit{Positive rotation is clockwise}.
    
    \begin{figure}[h]
        \centering
        \includegraphics[width=0.5\textwidth]{figs/hw1_pedal.png}
        \caption{A simple model of a control pedal of an aircraft.}
        \label{fig:pedal}
    \end{figure}
    
    
    \begin{enumerate}[label=(\alph*)]
        \item What is the equation of motion describing $\theta$? And what is the corresponding natural frequency $\omega_n$? Assume small angle approximation. Do not neglect gravity. Recall that the moment of inertia for a point mass with mass $m$ at a distance $\ell$ away from the axis of rotation is $m\ell^2$. \\\\
        Looking at the given system, only two forces act on the hinged beam on a given rotation of the pedal. $F_{spring} = k\Delta$ acting to the left on segment $l_1$. $F_{gravity} = mg$ acting on the pedal on segment $l_2$. The magnitude of force delivered by the spring is given by $F_{spring} = k\theta l_1$, since total displacement of the spring is
        the arc length of the rotation of the beam. Because we are looking for the equation of motion of $\theta$, we need to relate these forces using the moment about the hinge. Because $\theta$ is positive clockwise, the moment is positive clockwise. Hence,
        \[ \Sigma M = I\ddot{\theta} = ml^2_2\ddot{\theta} = -mgl_2\sin(\theta) - kl^2_1\theta\cos(\theta) \] Applying the small angle approximation $\cos(\theta) \approx 1, \sin(\theta) \approx \theta$, and moving terms to the other side, we get the equation of motion describing $\theta$. 
        \[ \Longrightarrow ml^2_2\ddot{\theta} + mgl_2\theta + kl^2_1\theta = ml^2_2\ddot{\theta} + \left(mgl_2+ kl^2_1\right)\theta = 0 \] we can solve for the natural frequency $w_n$ by treating this system as an MSD (Mass, Spring, Damper) system. For an MSD system $w_n = \sqrt{\frac{k}{m}}$. Hence the natrual frequency of this system is given by
        \[ \Longrightarrow w_n = \sqrt{\frac{k}{m}} = \sqrt{\frac{mgl_2 + kl_1^2}{ml_2^2}} \]

    \end{enumerate}

    
    \newpage 
    \item \textbf{(Modeling and natural frequency.)}
    Consider an aircraft wing with a fuel pod mounted on its tip. An illustration of the wing and fuel pod is shown in Figure~\ref{fig:aircraft wing and fuel pod}. The pod has a mass of 20kg when it is empty, and 2000kg when it is full. Calculate the change in the natural frequency of vibration of the wing as the aircraft uses up the fuel in the wing pod (from full to empty). The estimated physical parameters of the beam are: $I=5.2 \times 10^{-5}$m$^4$, $E=6.9 \times 10^9$Nm$^{-2}$, and $\ell=2$m.

    Are there any issues that you may need to consider as an engineer working on other components of this aircraft?

    \begin{figure}[h]
    \centering
        \includegraphics[width=0.9\textwidth]{figs/hw1_aircraft.png}
        \caption{Figure of aircraft and field pod on its wing.}
        \label{fig:aircraft wing and fuel pod}
    \end{figure}

    From lecture, we know that the approximate spring constant of a beam with a point mass on the end is given by $\frac{3EI}{l^3}$. \\
    \[ \Sigma F = m \ddot{x} = mg - kx \longrightarrow m\ddot{x} + kx = mg \] Treating the wing as an MSD system, the natural frequency $w_n$ is given by $\sqrt{\frac{k}{m}}$. Hence
    \[ w_n = \sqrt{\frac{k}{m}} = \sqrt{\frac{3EI}{l^3m}} \] Because only the mass changes during the flight \[ \Delta w_n = \sqrt{\frac{3EI}{l^3\Delta m}} = \sqrt{\frac{3EI}{l^3}} \left(\frac{1}{\sqrt{m_{empty}}} - \frac{1}{\sqrt{m_{full}}}\right) \] 
    Hence
    \[ \Longrightarrow \Delta w_n = \sqrt{\frac{3(6.9 * 10^9 \left[\frac{N}{m^2}\right])(5.2*10^{-5}\left[m^4\right])}{(2 \left[m\right])^3}}\left(\frac{1}{\sqrt{200 \left[kg\right]}} - \frac{1}{\sqrt{2000 \left[kg \right]}}\right) = 17.74 \left[\frac{rad}{s}\right] \] 
    Or $2.82 \left[Hz\right]$. Other engineers working on this aircraft need to be aware of this as to not contstruct a system that has a resonant frequency between the range of the natural frequencies the wing has throughout the flight. Doing this could lead to uncontrollable vibrations due to said resonance in the given system
    and in result cause catastrophic failure of the plane.

    \newpage
    \item \textbf{(Past exam problem.)}
    Your engineering firm is tasked with designing the suspension system of a moon lander spacecraft, shown in Figure~\ref{fig:spacecraft}. The suspension system consists of two \textit{identical} spring-damper systems.
    The spacecraft is carrying an extremely delicate payload that cannot experience any oscillation whatsoever otherwise it will break. Your task as the vibrations engineer in the firm is to design the suspension system to ensure no oscillations occur during landing.
    
    Assume that the landing pads touch the ground simultaneously and once they touch the ground, they remain rigidly attached to the ground and will not move or slip. Zero vertical displacement of the spacecraft ($y=0$) is defined when the spacecraft is just about to touch the ground and the suspension system is not yet compressed.
    
    
    \begin{figure}[h]
        \centering
        \includegraphics[width=0.6\textwidth]{figs/hw1_spacecraft.png}
        \caption{Moon lander spacecraft with a double suspension system.}
        \label{fig:spacecraft}
    \end{figure}
    
    \begin{enumerate}[label=(\alph*)]
        \item What is the equation of motion describing the vertical displacement $y$ of the spacecraft right after the pads make contact with the ground? 
        Do not neglect acceleration due to gravity $g\,\mathrm{ms}^{-2}$. \\\\
        Because the systems are identical, and are in parallel, the dampers and springs can be added together to form one damper and spring system with damper and spring constant 2b and 2k respectively. 
        With this in mind, we can take a sum of forces on the body to find the equation of motion.
        \[ \Sigma F = m\ddot{y} = mg - 2ky - 2b\dot{y} \] 
        With some rearrangement, we find the equation of motion to be 
        \[ \Longrightarrow m\ddot{y} + 2b\dot{y} + 2ky = mg \]

        \pagebreak
    
        \item Let $\dot{y}_0$ denote the vertical velocity of the spacecraft right before it touches the surface. If $Y(s)$ is the Laplace transform of $y(t)$, i.e., $Y(s) = \mathcal{L}\{y(t)\}$, then show that 
        \[Y(s) = \dfrac{1}{s}\cdot\dfrac{g}{s^2 + \frac{2b}{m}s + \frac{2k}{m}} + \dfrac{\dot{y}_0}{s^2 + \frac{2b}{m}s + \frac{2k}{m}}.\] \\
        Start by taking the Laplace transform of the equation of motion
        \[ \Lagr\left\{mg\right\} = m\Lagr\left\{\ddot{y}\right\} + 2b\Lagr\left\{\dot{y}\right\} + 2k\Lagr\left\{y\right\} \] 
        \[ \frac{mg}{s} = m(s^2Y(s) - sy(0) - \dot{y}_0) + 2b(sY(s) - y(0)) + 2kY(s) \]
        Plug in initial conditions and rearrange
        \[ \frac{mg}{s} = Y(s)(ms^2 + 2bs + 2k) - m\dot{y}_0 \longrightarrow \frac{1}{s}g + \dot{y}_0 = Y(s)(s^2 + \frac{2b}{m}s + \frac{2k}{m}) \]
        \[ \Longrightarrow \frac{1}{s} * \frac{g}{s^2 + \frac{2b}{m}s + \frac{2k}{m}} + \frac{\dot{y}_0}{s^2 + \frac{2b}{m}s + \frac{2k}{m}} \]

        \item For the spacecraft to experience \textit{no oscillations} when it lands, what are some constraints on $b$, $m$, and $k$? \\\\
        In order to have zero oscillations upon landing, the system must be at least critically damped, as to say that the damping ratio ($\zeta$) must be $\geq 1$ \\ 

        We can model the spacecrafts suspension as an MSD system, and therefore the damping ratio is given by $\zeta = \frac{c}{2\sqrt{km}} = \frac{2b}{2m\sqrt{\frac{2k}{m}}}$. Which simplifies to the inequality
        \[ \Longrightarrow \zeta = \frac{b}{\sqrt{2km}} \geq 1 \]

        \pagebreak
    
        \item You find that the mass of the spacecraft is 10kg (it is a small spacecraft!) and that the only springs and dampers that are certified to operate in space have coefficients $k=3\,\mathrm{kgs}^{-2}$ and $b=4\,\mathrm{kgs}^{-1}$. Unfortunately, you find that these spring and damping coefficients lead to the spacecraft oscillating. But you decide to get creative and create ``new'' springs by combining \textit{two} springs either in series $k_\mathrm{s}$ or in parallel $k_\mathrm{p}$. Similarly, you can create ``new'' dampers by combining \textit{two} dampers either in series $b_\mathrm{s}$ or in parallel $b_\mathrm{p}$.
        With more spring and damper options available, is there a feasible design that prevents the spacecraft from oscillating? If no, write no. If yes, from the set of possible combinations illustrated below, circle which combination(s) is(are) feasible. Briefly justify your answer.


        
        \textbf{Remark 1}: Both suspension systems must be identical and use the same spring and damper combination.     
        \textbf{Remark 2}: In the figures below, only one of the suspensions is illustrated.    
        \textbf{Remark 3}: Placing dampers in parallel/series follows the same formulas as placing springs in parallel/series.
    \end{enumerate}
    
    \begin{figure}[h]
        \centering
        \begin{subfigure}{0.22\textwidth}
          \centering
          \includegraphics[width=\linewidth]{figs/bkp.png}
          \caption{Single damper, parallel spring. ($b, k_p$)}
          \label{fig:bkp}
        \end{subfigure}%
        \hfill
        \begin{subfigure}{0.22\textwidth}
          \centering
          \includegraphics[width=\linewidth]{figs/bks.png}
          \caption{Single damper, series spring. ($b, k_s$)}
          \label{fig:bks}
        \end{subfigure}%
        \hfill
        \begin{subfigure}{0.22\textwidth}
          \centering
          \includegraphics[width=\linewidth]{figs/bpk.png}
          \caption{Parallel damper, single spring. ($b_p, k$)}
          \label{fig:bpk}
        \end{subfigure}%
        \hfill
        \begin{subfigure}{0.22\textwidth}
          \centering
          \includegraphics[width=\linewidth]{figs/bsk.png}
          \caption{Series damper, single spring. ($b_s, k$)}
          \label{fig:bsk}
        \end{subfigure}\\
        \vspace{10mm}
        \begin{subfigure}{0.22\textwidth}
          \centering
          \includegraphics[width=\linewidth]{figs/bpkp.png}
          \caption{Parallel damper, parallel spring. ($b_p, k_p$)}
          \label{fig:bpkp}
        \end{subfigure}%
        \hfill
        \begin{subfigure}{0.22\textwidth}
          \centering
          \includegraphics[width=\linewidth]{figs/bpks.png}
          \caption{Parallel damper, series spring. ($b_p, k_s$)}
          \label{fig:bpks}
        \end{subfigure}%
        \hfill
        \begin{subfigure}{0.22\textwidth}
          \centering
          \includegraphics[width=\linewidth]{figs/bskp.png}
          \caption{Series damper, parallel spring. ($b_s, k_p$)}
          \label{fig:bskp}
        \end{subfigure}%
        \hfill
        \begin{subfigure}{0.22\textwidth}
          \centering
          \includegraphics[width=\linewidth]{figs/bsks.png}
          \caption{Series damper, series spring. ($b_s, k_s$)}
          \label{fig:bsks}
        \end{subfigure}
        \vspace{2mm}
        % \caption{The research goal is to develop ego-centric evaluation metrics for determining how effectively autonomous mobile robots navigate safely and fluently in dense human crowds.}
        \label{fig:key figures}
    \end{figure}
    
        First we can identify the equivalent spring and damper constants of both the series and parallel variations.
        \[k_p = 2k = 6 \left[\frac{kg}{s^2}\right] \quad k_s = \frac{1}{\frac{2}{k}} = \frac{k}{2} = \frac{3}{2} \left[\frac{kg}{s^2}\right] \]
        \[b_p = 2b = 8 \left[\frac{kg}{s}\right] \quad b_s = \frac{b}{2} = 2 \left[\frac{kg}{s}\right] \] \\
        From here it is as simple as checking each case to see if it fulfills the criteria found in the previous question. Instead of punching all of the numbers into a calculator I wrote a python script to do it instead. 
        The condition is still \[ \frac{b}{\sqrt{2km}} \geq 1 \]. 

        \begin{verbatim}
m = 10
kp = 6
ks = 3/2
bp = 8
bs = 2
k = 3
b = 4

options = [
# k, b, option letter
(kp, b, ’a’),
(ks, b, ’b’),
(k, bp, ’c’),
(k, bs, ’d’),
(kp, bp, ’e’),
(ks, bp, ’f’),
(kp, bs, ’g’),
(ks, bs, ’h’)
]

def condition(K, B): return B/math.sqrt(2*K*m)

print("Successful Options:")
for option in options:
  if condition(option[0], option[1]) >= 1: print(f" {option[2]}")

OUTPUT:
Successful Options:
c
f 
        \end{verbatim}
        As can be seen from the file output, the options that would work for the given MSD systems are options \textbf{C \& F}.

    
    \newpage 
    \item \textbf{(Optional, ungraded)} Prove the following Laplace identities.
        \begin{enumerate}[label=(\alph*)]
            
            \item $\mathcal{L}[f(ct)] = \frac{1}{c}F(\frac{s}{c})$
            
            \[ \Lagr\left\{f(ct)\right\} = \int_{0}^{\infty} e^{-st} f(ct) dt \]
            Take $u = ct \longrightarrow t = \frac{u}{c}$
            \[ \frac{1}{c} \int_{0}^{\infty} e^{\frac{-su}{c}} f(u) du = \frac{1}{c} \int_{0}^{\infty} e^{-\left(\frac{s}{c}\right)u} f(u) du \]
            \[ \Longrightarrow \frac{1}{c}F\left(\frac{s}{c}\right)\] 

            \item $\mathcal{L}[\int_0^t f(v)dv] = \frac{F(s)}{s}$. Hint: Let $g(t) = \int_0^t f(v)dv$ and $g^\prime(t) = f(t)$ (by the Fundamental Theorem of Calculus). \\\\
            \[ g(t) = \int_{0}^{t} f(v) dv \quad \& \quad \frac{dg}{dt} = f(t) \]
            \[ \Lagr\left\{\frac{dg}{dt}\right\} = sG(s) - g(0) = sG(s) - \int_{0}^{0} f(v)dv \]
            \[ \Longrightarrow \frac{\Lagr\left\{f(t)\right\}}{s} = \frac{F(s)}{s} = G(s) = \Lagr\left\{\int_{0}^{t} f(v) dv\right\} \]
\end{enumerate}
\end{enumerate}


\end{document}


